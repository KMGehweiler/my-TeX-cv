%%%%%%%%%%%%%%%%%%%%%%%%%%%%%%%%%%%%%%%%%
% Plasmati Graduate CV
% LaTeX Template
% Version 1.0 (24/3/13)
%
% This template has been downloaded from:
% http://www.LaTeXTemplates.com
%
% Original author:
% Alessandro Plasmati (alessandro.plasmati@gmail.com)
%
% License:
% CC BY-NC-SA 3.0 (http://creativecommons.org/licenses/by-nc-sa/3.0/)
%
% Important note:
% This template needs to be compiled with XeLaTeX.
% The main document font is called Fontin and can be downloaded for free
% from here: http://www.exljbris.com/fontin.html
%
%%%%%%%%%%%%%%%%%%%%%%%%%%%%%%%%%%%%%%%%%

%----------------------------------------------------------------------------------------
%	PACKAGES AND OTHER DOCUMENT CONFIGURATIONS
%----------------------------------------------------------------------------------------

\documentclass[a4paper,10pt]{article} % Default font size and paper size

\usepackage[ngerman]{babel}
\usepackage[utf8]{inputenc}

%----------------------------------------------------------------------------------------
%	FONT
%----------------------------------------------------------------------------------------
\usepackage[T1]{fontenc}

\usepackage{fontspec} % For loading fonts
%\defaultfontfeatures{Mapping=tex-text}
%\setmainfont[SmallCapsFont = Fontin SmallCaps]{Fontin} % Main document font

\usepackage[11pt]{moresize} %enables \HUGE

%----------------------------------------------------------------------------------------
%	LAYOUT
%----------------------------------------------------------------------------------------
%\usepackage[big]{layaureo} % Margin formatting of the A4 page, an alternative to layaureo can be 

%\usepackage{fullpage}
% To reduce the height of the top margin uncomment: 
%\addtolength{\voffset}{-1.3cm}

\usepackage[left=2cm,right=2cm,top=1cm,bottom=0cm,includeheadfoot]{geometry}

\usepackage{multirow}

\usepackage{array}
\newcolumntype{L}[1]{>{\raggedright\let\newline\\\arraybackslash\hspace{0pt}}m{#1}}
\newcolumntype{C}[1]{>{\rc\let\newline\\\arraybackslash\hspace{0pt}}m{#1}}
\newcolumntype{R}[1]{>{\raggedleft\let\newline\\\arraybackslash\hspace{0pt}}m{#1}}
%\renewcommand{\arraystretch}{0.9}

\usepackage{titlesec} % Used to customize the \section command
\titleformat{\section}{\Large\scshape\raggedright}{}{0em}{}[\titlerule] % Text formatting of sections
\titlespacing*{\section}{0pt}{1.2\baselineskip}{0.5\baselineskip}

\titleformat{\subsection}{\Large\scshape\raggedright}{}{0em}{}[] % Text formatting of sections
\titlespacing*{\subsection}{0pt}{3\baselineskip}{1.5\baselineskip}

\usepackage{enumitem}
\setlist{leftmargin=*, noitemsep}

\usepackage[rflt]{floatflt}

%----------------------------------------------------------------------------------------
%	COLOUR
%----------------------------------------------------------------------------------------
\usepackage[usenames,dvipsnames]{xcolor} % Required for specifying custom colors

\usepackage{hyperref} % Required for adding links	and customizing them
\definecolor{linkcolour}{rgb}{0,0,0} % Link color
\hypersetup{colorlinks,breaklinks,urlcolor=linkcolour,linkcolor=linkcolour} % Set link colors throughout the document

\definecolor{starblue}{RGB}{3,124,181}
\DeclareTextFontCommand{\emph}{\bfseries\color{starblue}}

%----------------------------------------------------------------------------------------
%	MISC
%----------------------------------------------------------------------------------------
\usepackage{picins} % Bilder positionieren
\usepackage{booktabs}
\usepackage{hanging}
\usepackage{lipsum}
\usepackage{xunicode,xltxtra,url,parskip} % Formatting packages
\usepackage{pdfpages}

%----------------------------------------------------------------------------------------
%	COMMANDS
%----------------------------------------------------------------------------------------
\newcommand{\rc}[1]{\multirow{2}{\linewidth}{\raggedleft #1}}
\newcommand{\ic}[1]{\multirow{}{\linewidth}{\begin{tabular}{@{}c@{}}
			#1
		\end{tabular}}}
		
\newcommand{\tabitem}{\footnotesize\hangindent=1.1em\hangafter=1\hspace{0.7em}\llap{\textbullet}\hspace{0.4em}}


%----------------------------------------------------------------------------------------
%	-------------------------------------------------------------------------------------
%----------------------------------------------------------------------------------------
\begin{document}

	
\pagestyle{empty} % Removes page numbering

	
\begin{center}
\vspace*{3em}
\textsc{\HUGE Bewerbung}\\[2em]
\textsc{\huge im Bereich Datenschutz}\\[5em]


\newcommand{\HRule}{\rule{\linewidth}{0.5mm}}
\HRule \\[3em]

\includegraphics[width=0.6\linewidth]{Gehweiler_Karin} 
\vspace{3em}

\vspace{6em}
\LARGE
\textsc{Karin M Gehweiler}\\ \vspace{2em}

\large
Brühlweg 4\\
88499 Riedlingen\\
\vspace{1em}
Telefon: 0173 267 23 16\\
eMail: karin.gehweiler@gmail.com\\

\vspace{3em}
\hrule 
\vspace{1em}
\Large
\textsc{Inhalt}\\
\vspace{0.5em}
\large
\textbullet Lebenslauf \textbullet Empfehlungsschreiben
\textbullet Notenübersicht \textbullet Diplom-Zeugnis




\end{center}
\newpage


%----------------------------------------------------------------------------------------
%	-------------------------------------------------------------------------------------
%----------------------------------------------------------------------------------------



\pagestyle{empty} % Removes page numbering

%%----------------------------------------------------------------------------------------
%%	ANSCHREIBEN
%%----------------------------------------------------------------------------------------
%
%\par{\Huge \textsc{Karin M Gehweiler}\par}
%\hspace{40pt} Brühlweg 4; 88499 Riedlingen | 0173-2672316 | karin.gehweiler@gmail.com\\
%
%
%\vspace{50pt}
%Audi AG\\
%85045 Ingolstadt
%\vspace{20pt}
%
%{\hfill31.07.2014}
%
%
%\subsection{Bewerbung als Datenanalyst}
%
%Sehr geehrte Damen und Herren, 
%
%\vspace{10pt}
%hiermit bewerbe ich mich auf eine Stelle als Datenanalyst in Ihrem Unternehmen. 
%
%Ich habe letztes Jahr mein Studium "Diplom-Mathematik" an der Universität Augsburg abgeschlossen und glaube, dass gerade in der Automobil-Branche in jedem Schritt von der Entwicklung bis zum Verkauf einige Daten anfallen, welche sehr aufschlussreich sind, um die Produktion zu sichern und zu verbessern.
%
%Ich habe großen Spaß an der  Arbeit mit Daten  und deren Visualisierung  und konnte schon einige
%Erfahrungen  in  diesem Bereich  sammeln.  Hierzu  gehören  zahlreiche  praktische  Arbeiten  mit
%Datensätzen und statistischen Verfahren während meines Studiums,  meiner Mitarbeit am R-Paket
%„extracat“ (statische Parallel-Koordinaten-Plots für kategorielle Daten) und meiner Mitarbeit an der
%Überarbeitung  der  mobilen  Version  des  Risiko-Visualisierungstools  „Riskroom“  für  Zürich
%Versicherungen bei anfema GmbH.
%
%
%Zu mir:
%
%Mein Profil  ist  vielseitig. Einerseits biete ich  durch mein Studium eine fundierte Basis  im Bereich
%Statistik und Datenanalyse. Andererseits lehrten mich meine verschiedenen Tätigkeiten in Nebenjobs
%und Ehrenämtern die Wichtigkeit von Gewissenhaftigkeit, zielorientiertem Handeln und nicht zuletzt
%die Arbeit im Team. Darüber hinaus versteht es sich für mich als selbstverständlich, stets neues lernen
%zu wollen und mich weiterzubilden.
%
%Ich  glaube,  dass Sie  sich  ein  umfassendes  Bild  von  mir  durch  meinen  Lebenslauf,  meine
%Notenübersicht und den beigelegten Empfehlungsschreiben machen können. 
%
%Falls Sie noch weitere Fragen haben oder  einen Termin für  ein Vorstellungsgespräch vereinbaren
%möchten, erreichen Sie mich am besten per Mail unter karin.gehweiler@gmail.com. 
%
%
%Ich freue mich auf Ihre Antwort.
%
%\vspace{20pt}
%Mit freundlichen Grüßen 
%
%\vspace{20pt}
%\includegraphics[width=0.33\linewidth]{unterschrift_03}
%
%Karin M Gehweiler 
%
%
%\newpage




%\font\fb=''[cmr10]'' % Change the font of the \LaTeX command under the skills section

%----------------------------------------------------------------------------------------
%	NAME AND CONTACT INFORMATION
%----------------------------------------------------------------------------------------}
%\parpic[rf]{\includegraphics[width=0.33\linewidth]{Gehweiler_Karin}}


\par{\Huge \textsc{Karin M \\Gehweiler}\bigskip\par} % Your name


%\vspace{10pt}
\section{Persönliche Daten}

\begin{tabular}{R{0.2\linewidth}p{0.76\linewidth}}
\textsc{Adresse:} & Brühlweg 4, 88499 Riedlingen \\
\textsc{eMail:} &\href{mailto:karin.gehweiler@gmail.com}{karin.gehweiler@gmail.com}\\
\textsc{Telefon:} & 0173 267 23 16\\
\textsc{Geburtstag:} & 23. Dezember 1985\\
\textsc{Familienstand:} & ledig\\
\end{tabular}

%----------------------------------------------------------------------------------------
%	WORK EXPERIENCE 
%----------------------------------------------------------------------------------------

\section{Berufserfahrung}

\begin{tabular}{R{0.2\linewidth}|p{0.755\linewidth}}

\rc{09/2014 - jetzt} & \textsc{comforts counsulting GmbH} \\&\emph{Festangestellt} \\
&   \tabitem Projektarbeit: Überarbeitung und Optimierung OTRS-HelpDesk \\

\multicolumn{2}{c}{} \\

%------------------------------------------------


\rc{10/2013 - 08/2014} & \textsc{elan GmbH} \& \textsc{Radke Imbissbetrieb} \\&\emph{Reinigungs- und Verkaufskraft} \\ 
\multicolumn{2}{c}{} \\

%------------------------------------------------

\rc{04/2013 - 05/2013} & \textsc{anfema GmbH} \\&\emph{Praktikantin} \\
&   \tabitem Implementierung einer automatisierten Übersetzung und Code-CleanUp für ein Web-Portal \\
&   \tabitem Beratung und Kontroll-Rechnungen für die Weiterentwicklung einer App zur Visualisierung statistischer Daten \\

\multicolumn{2}{c}{} \\

%------------------------------------------------

\rc{07/2010 - 03/2013} & \textsc{Universität Augsburg - Lehrstuhl für rechnerorientierte Statistik und Datenanalyse} \\&\emph{Studentische Hilfskraft (Sekretariat)}\\
& 	\tabitem Buchhaltung und generelle organisatorische Aufgaben am Lehrstuhl\\
&	\tabitem Organisation eines Workshops zum Thema R als berufliche Weiterbildung\\
\multicolumn{2}{c}{} \\

%------------------------------------------------

\rc{09/2006 - 06/2010} & \textsc{Universität Augsburg - Rechenzentrum} \\&\emph{Studentische Hilfskraft (Management \& Support)}\\
& 	\tabitem Support für Studenten bzgl. der universitären IT-Infrastruktur (z.B. WLAN, Computerräume, Drucken, ...)\\
&	\tabitem Support für Studenten bei Problemen mit ihrem Notebooks (z.B. Virenbefall, ...)\\
&	\tabitem Wartung und Pflege der Computerräume des Rechenzentrums\\

\multicolumn{2}{c}{} \\

%------------------------------------------------

\rc{SS 2010 - WS 2012} & \textsc{Universität Augsburg - Institut für Informatik} \\&\emph{Studentische Hilfskraft (Administration der Lehrevalation)}\\
& 	\tabitem Einpflegen der jeweils aktuellen Veranstaltungs-Daten\\
&	\tabitem Koordination des Ablaufs der Evaluation\\
	
\end{tabular}

%----------------------------------------------------------------------------------------
%	LANGUAGES
%----------------------------------------------------------------------------------------

\section{Fachkenntnisse}


\begin{tabular}{R{0.15\linewidth}p{0.25\linewidth}p{0.075\linewidth}R{0.15\linewidth}L{0.25\linewidth}}
	\multicolumn{2}{c}{Programmiersprachen \& Programme} & & \multicolumn{2}{c}{Sprachen} \\
	\cline{1-2} \cline{4-5} \\
	R: & \includegraphics[width=5em]{4stars} && \textsc{Deutsch:} & Muttersprache \\
	\LaTeX: & \includegraphics[width=5em]{4stars} && \textsc{Englisch:} & Fliesend\\
	Office: &\includegraphics[width=5em]{3stars} && \textsc{Französisch:} & Grundkenntnisse \\
	HTML/CSS: & \includegraphics[width=5em]{3stars}&&&\\
	Weitere: & SQL, Django, Java&&&\\ %, CMS (Fiona), Wordpress \\
\end{tabular}
%\hfill
%\begin{tabular}{}
	%\multicolumn{2}{c}{Sprachen}\\
	%\cline{1-2}\\
	%\textsc{Deutsch:} & Muttersprache \\
	%\textsc{Englisch:} & Fliesend\\
	%\textsc{Französisch:} & Grundkenntnisse\\
	%\\ \\ \\ \\
%\end{tabular}
		
%----------------------------------------------------------------------------------------
%	EDUCATION
%----------------------------------------------------------------------------------------
\section{Bildung}

\begin{tabular}{R{0.2\linewidth}p{0.76\linewidth}}
2006 - 2013 & \textsc{Universität Augsburg}\\
& \emph{Mathematik mit Nebenfach Informatik}\\
& Vertiefung Statistik \& Datenanalyse\\
&\normalsize \textsc{Abschluss}: \emph{Diplom}, Note 2,8\\ % \hyperlink{grds}{\hfill | \footnotesize Detaillierte Notenliste}\\ %\hyperlink{dzeugnis}{~| \footnotesize Zeugnis}\\
&\\
%------------------------------------------------

2005 - 2006 & \textsc{Universität Augsburg}\\
& Mathematik auf Gymnasiallehramt mit Nebenfach Physik\\
&\\
%------------------------------------------------

\textsc{2005} & \textsc{Kreisgymnasium Riedlingen}\\
& \textsc{Abschluss}: Allgemeine Hochschulreife, Note 2,1 \\ %\hyperlink{azeugnis}{\hfill| \footnotesize Zeugnis}\\
&\\
\end{tabular}

%----------------------------------------------------------------------------------------
%	SCHOLARSHIPS AND ADDITIONAL INFO
%----------------------------------------------------------------------------------------

%\section{Scholarships and Certificates}
%
%\begin{tabular}{rl}
%\textsc{Sept.} 2012 & Faculty of Science Masters Scholarship \footnotesize(\$30,000)\normalsize\\
%
%\textsc{June} 2010 & {\textsc{Gmat}\textregistered}\setmainfont[SmallCapsFont=Fontin SmallCaps]{Fontin-Regular}: 730 (\textsc{q:50;v:39}) 96\textsuperscript{th} percentile; \textsc{awa}: 6.0/6.0 (89\textsuperscript{th} percentile)
%\end{tabular}

%----------------------------------------------------------------------------------------
%	WORK EXPERIENCE 
%----------------------------------------------------------------------------------------

\section{Persönliches Engagement}

\begin{tabular}{R{0.2\linewidth}|p{0.75\linewidth}}
	\rc{10/2011 - 03/2013} & \textsc{Universität Augsburg} \\&\emph{Mitglied im Ältestenrat} \\
	& \tabitem Beratung, Schlichtung und Kontrollinstanz für alle Organe der studentischen Mitbestimmung\\
	\multicolumn{2}{c}{} \\
	
	%------------------------------------------------
	
	\rc{10/2010 - 09/2011} & \textsc{Universität Augsburg} \\&\emph{Vizepräsidentin im Studentischen Konvent} \\
	&   \tabitem Organisation und Protokollierung der Sitzungen \\
	\multicolumn{2}{c}{} \\
	
	%------------------------------------------------
	
	\rc{02/2006 - 03/2013} & \textsc{Universität Augsburg}\\&\emph{Mitglied in der Fachschaft Mathematik}\\
	& 	\tabitem Anlaufstelle für studentische Belange aller Art\\
	&	\tabitem Organisation und Durchführung von sozialen und studiumsrelevanten Maßnahmen und Veranstaltungen\\
	\multicolumn{2}{c}{} \\
	
	%------------------------------------------------
	
	\rc{10/2008 - 05/2009} & \textsc{Universität Augsburg}\\&\emph{Fachschaftssprecherin}\\
	&	\tabitem Entscheidungen über materielle und personelle Maßnahmen zur Verbesserung der Studiensituation aus Studienbetragsgeldern\\
	&	\tabitem Leitung der Fachschaft Mathematik\\
	&	\tabitem Mitbestimmung auf Instituts- \& Fakultätsebene\\
	&	\tabitem Organisation zweier internationaler Studierendenkonferenzen\\
	
\end{tabular}


%%----------------------------------------------------------------------------------------
%%	COMPUTER SKILLS 
%%----------------------------------------------------------------------------------------
%
%\section{Computer-Kenntnisse}
%
%\begin{tabular}{rl}
%Grundkenntnisse: & \textsc{}, my\textsc{sql}, \textsc{html}, Access, \textsc{Linux}, ubuntu, \\
%%
%%Fortgeschritten: & \textsc{R}, {\fb \LaTeX}\setmainfont[SmallCapsFont=Fontin SmallCaps]{Fontin-Regular}\\
%\end{tabular}

%----------------------------------------------------------------------------------------
%	INTERESTS AND ACTIVITIES
%----------------------------------------------------------------------------------------

\section{Weitere Informationen}

\begin{tabular}{R{0.2\linewidth}p{0.76\linewidth}}
	\textsc{Interessen:} & Go (spielen \& Kommentieren von Partien), Handarbeiten, Leitung eines Buchklubs\\
	&\\
	\textsc{Eigenschaften:} & zuverlässig und gewissenhaft mit ausgeprägter Organisationsstärke, Lösungsorientierung und hoher Lernbereitschaft
\end{tabular}
\vspace{1em}
{\parpic[r]{\includegraphics[width=0.2\linewidth]{unterschrift_03}}
\vspace{2em}
\hfill \today}
%----------------------------------------------------------------------------------------

\newpage

%----------------------------------------------------------------------------------------
%	EMPFEHLUNGEN
%----------------------------------------------------------------------------------------

\addcontentsline{toc}{section}{Empfehlungsschreiben Unwin}
\includepdf[scale=1]{Empfehlung_Unwin.pdf}

\addcontentsline{toc}{section}{Empfehlungsschreiben Jilg}
\includepdf[scale=1]{Empfehlung_Jilg.pdf}

%----------------------------------------------------------------------------------------

\newpage

%----------------------------------------------------------------------------------------
%	GRADE TABLES
%----------------------------------------------------------------------------------------
\vspace*{3em}
\par{\centering\Large \hypertarget{grds}{\textsc{Diplom-Mathematik}\\mit Nebenfach Informatik}\par}
\vspace{0.5em}\LARGE{\centering \textsc{Notenübersicht Hauptstudium}\par}\normalsize
\vspace{4em}
\begin{center}\large
\begin{tabular}{clc}
\multicolumn{1}{c}{\textsc{Typ}} & \textsc{Veranstaltung}&\textsc{Note}\\ \hline
\vspace{2pt}\\
S& Stochastische Prozesse& 1,3\\
V& Statistik 3& 1,3\\
V& Grafische Datenanalyse& 1,3\\
S& Seminar zu Algebra und Zahlentheorie& 1,7\\
V& Datenbanken& 1,7\\
S& Data Technologies& 1,7 \\
S& Charakteristische Klassen& 1,7\\
V& Datenstrukturen& 2\\
V& Statistik 1& 3,7\\
V& Gewöhnliche Differentialgleichungen& 3,7\\
V& Algebra& 3,7\\
V& Quaternionen und Oktaven& 4\\
V& Optimierung 2& 4\\
V& Informatik 3& 4\\	
%\\\cline{2-3}
%&\textsc{gesamt}&\textbf{8.0}
\end{tabular}
\end{center}

%----------------------------------------------------------------------------------------
%	DIPLOMSZEUGNISS
%----------------------------------------------------------------------------------------

\addcontentsline{toc}{section}{Diplom-Zeugnis}
\includepdf[scale=1, pages={1-2}]{Diplomszeugnis.pdf}


%----------------------------------------------------------------------------------------

\end{document}
